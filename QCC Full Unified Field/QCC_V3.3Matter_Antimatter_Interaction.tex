
\documentclass[12pt]{article}
\usepackage{amsmath, amssymb, graphicx}
\usepackage[margin=1in]{geometry}
\usepackage{physics}
\title{QCC Checkpoint: Matter-Antimatter Memory Interference and Constructive Collapse}
\author{Quantum Coherence Cosmology (QCC)}
\date{\today}

\begin{document}

\maketitle

\section*{Overview}
This document records a key checkpoint in Quantum Coherence Cosmology (QCC): the successful simulation of a matter-antimatter plasma memory field derived from Planck CMB harmonics. Surprisingly, rather than canceling, the interference between matter and antimatter coherence fields yielded a deeper bound state. This suggests a mechanism for neutral boson formation and supports the idea of antimatter as a memory echo from the initial cosmic singularity.

\section*{1. Initial Conditions}
\begin{itemize}
  \item Radial domain: $r \in [0.01, 10]$ with 1000 points
  \item Field source: FFT-decomposed harmonics from COM\_PCCS 030 GHz Planck data
  \item Selected harmonics: Top 3 energetic modes used to simulate plasma memory field
  \item Energy range: $E \in [-1000, 0]$
\end{itemize}

\section*{2. Plasma Memory Field Construction}
Using three dominant harmonic modes $\{k_1, k_2, k_3\}$, we constructed:
\[
\phi_\text{matter}(r) = \sum_{n=1}^3 A_n \sin(k_n r)
\]
where $A_n$ and $k_n = 2\pi f_n / 720$ are the amplitudes and spatial frequencies from the FFT of DETFLUX-binned Planck data. The function is centered and normalized as:
\[
\phi(r) \leftarrow \frac{\phi(r) - \langle \phi \rangle}{\max |\phi(r)|}
\]
The antimatter coherence field was defined by phase inversion:
\[
\phi_\text{antimatter}(r) = -\phi_\text{matter}(r)
\]
and normalized similarly.

\section*{3. Schr\"odinger-like Simulation of Bound States}
The wavefunctions were numerically integrated by solving the radial 1D time-independent Schr\"odinger-like equation:
\[
\frac{d^2 \psi(r)}{dr^2} = [E - \phi(r)] \psi(r)
\]
using finite difference methods:
\begin{align*}
\psi[0] &= 0 \\
\psi[1] &= \varepsilon \ll 1 \\
\psi[i] &= 2\psi[i-1] - \psi[i-2] + \Delta r^2 (E - \phi[i-1]) \psi[i-1]
\end{align*}
The solution $\psi(r)$ is normalized using Simpson integration:
\[
\psi(r) \leftarrow \frac{\psi(r)}{\sqrt{\int \psi^2 dr}}
\]
The optimal energy $E_\text{bound}$ is found by maximizing the peak amplitude:
\[
E_\text{bound} = \arg\max_E \left( \max_r |\psi(r)| \right)
\]
using scalar minimization with bounded search between $-1000$ and $0$.

\section*{4. Results}
\begin{itemize}
  \item \textbf{Matter field only:}
    \begin{itemize}
      \item $E = -382.03$, \quad $\text{MeV} \approx 0.466$
    \end{itemize}
  \item \textbf{Antimatter field only:}
    \begin{itemize}
      \item $E = -381.96$, \quad $\text{MeV} \approx 0.466$
    \end{itemize}
  \item \textbf{Combined field:}
    \begin{itemize}
      \item $E = -618.03$, \quad $\text{MeV} \approx 0.754$
    \end{itemize}
\end{itemize}

\section*{5. Interpretation}
Contrary to expectation, the combined field $\phi_\text{matter} + \phi_\text{antimatter}$ did not cancel to zero. Instead, it produced a \textbf{deeper bound state}, indicating:
\begin{itemize}
  \item Memory interference in the early universe was \textbf{constructive}, not annihilative
  \item The resulting coherence trap may correspond to a \textbf{neutral boson} (photon or $Z^0$)
  \item Antimatter is best interpreted in QCC as a \textbf{phase-inverted memory echo} from the singularity
\end{itemize}

\section*{6. Implications and Next Directions}
\begin{itemize}
  \item Explore phase-offset and drifted harmonics for spinor generation
  \item Simulate positronium-like bound states via loose memory interference
  \item Investigate deeper binding and decay structure using higher harmonic interference
\end{itemize}

\end{document}
