
\documentclass[12pt]{article}
\usepackage{amsmath, amssymb, graphicx}
\usepackage[margin=1in]{geometry}
\usepackage{physics}
\title{QCC Checkpoint: Matter-Antimatter Memory Interference and Constructive Collapse}
\author{Quantum Coherence Cosmology (QCC)}
\date{\today}

\begin{document}

\maketitle

\section*{Overview}
This document records a key checkpoint in Quantum Coherence Cosmology (QCC): the successful simulation of a matter-antimatter plasma memory field derived from Planck CMB harmonics. Surprisingly, rather than canceling, the interference between matter and antimatter coherence fields yielded a deeper bound state. This suggests a mechanism for neutral boson formation and supports the idea of antimatter as a memory echo from the initial cosmic singularity.

\section*{1. Initial Conditions}
\begin{itemize}
  \item Radial domain: r in [0.01, 10.0] with 1000 points
  \item Field source: FFT-decomposed harmonics from COM\_PCCS 030 GHz Planck data
  \item Selected harmonics: Top 3 energetic modes used to simulate plasma memory field
  \item Energy range: E in [-1000, 0]
\end{itemize}

\section*{2. Plasma Memory Field Construction}

Using three dominant harmonic modes $k_1$, $k_2$, and $k_3$, we constructed:

\[
	\phi_{\text{matter}}(r) = A_1 \sin(k_1 r) + A_2 \sin(k_2 r) + A_3 \sin(k_3 r)
\]

where $k = \frac{2\pi \cdot \text{frequency}}{720}$, and amplitudes $A$ come from the FFT of binned \texttt{DETFLUX} Planck data.

The antimatter coherence field is defined as:

\[
	\phi_{\text{antimatter}}(r) = -\phi_{\text{matter}}(r)
\]

For constructive recombination simulations:

\[
	\phi_1(r) = A_1 \sin(k_1 r)
\]
\[
	\phi_2(r) = A_2 \sin\left(k_2 (r - \Delta r)\right)
\]
\[
	\phi_{\text{recombination}}(r) = \phi_1(r) + \phi_2(r)
\]

	Where $\Delta r$ is the spatial offset simulating particle proximity during recombination.

For photon emission simulation:

\[
	\phi_{\text{collapsed}}(r) = \phi_{\text{recombination}}(r) \cdot \left(1 - \exp\left[-5(r - 5)^2\right]\right)
\]

\section*{3. Wavefunction and Bound Energy Simulation}
We solve the time-independent Schrödinger-like equation:

\[
	\frac{d^2 \psi}{dr^2} = \left( E - \phi(r) \right) \psi(r)
\]

With initial conditions:

\[
	\psi[0] = 0
\]
\[
	\psi[1] = \epsilon \quad \text{(small value, e.g., } 10^{-5} \text{)}
\]
\[
	\psi[n] = 2 \psi[n-1] - \psi[n-2] + \Delta r^2 \cdot \left( E - \phi[n-1] \right) \psi[n-1]
\]

Normalization:

\[
	\psi_{\text{normalized}} = \frac{\psi}{\sqrt{\int \psi^2 \, dr}}
\]

The bound state energy is found by optimizing:

\[
	\text{maximize } \left| \psi \right|
\]

across $E$ in the specified domain.


\section*{4. Results}
\begin{itemize}
  \item Electron/positron state:
    \begin{itemize}
      \item E = -382.03, MeV = 0.466
    \end{itemize}
  \item Recombination raw amplitude:
    \begin{itemize}
      \item E = -9.35, MeV = 0.0114
    \end{itemize}
  \item Collapsed recombination (photon emission):
    \begin{itemize}
      \item E = -425.60, MeV = 0.519
    \end{itemize}
\end{itemize}

\section*{5. Interpretation}
\begin{itemize}
  \item Normalization was masking true energetic variation of coherence traps
  \item Raw harmonic amplitude yields physically correct shallow traps pre-collapse
  \item Collapse induces deeper energy resonance — matching expected photon energy from recombination
  \item This simulates emission from a neutral hydrogen formation event in early universe QCC
\end{itemize}

\section*{6. Color Loop Construction}
To simulate QCD-like behavior:
\begin{itemize}
  \item Three phase-shifted coherence fields constructed:
    \begin{itemize}
      \item phi1 = A1 * sin(k1 * r)
      \item phi2 = A2 * sin(k2 * r + 2pi/3)
      \item phi3 = A3 * sin(k3 * r + 4pi/3)
    \end{itemize}
	\item Summed into: $\phi_{\text{loop}}(r) = \phi_1 + \phi_2 + \phi_3$
	\item Resulting bound state: $E = -0.683$, MeV $\approx 0.00083$
	\item Interpretation: pre-quark color coherence node

\end{itemize}

\section*{7. Spin Loop Interactions}
Spin coupling simulated via field overlays:
\begin{itemize}
  \item Phase rotation (pi/4) applied across loop triplets
  \item Resulting spin-coupled wavefunction:
	\begin{itemize}
		\item $\psi_{\text{spin\_coupled}} = \psi_{\text{A\_spin}} + \psi_{\text{B\_spin}} + \psi_{\text{C\_spin}}$
	\end{itemize}
  \item Forms coherent baryon-like structure
  \item Matches QCC prediction: stable spin-locked color triplets emerge naturally from memory field harmonics
\end{itemize}

\end{document}
