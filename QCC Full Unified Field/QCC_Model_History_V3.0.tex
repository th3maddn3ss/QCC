
\documentclass[12pt]{article}
\usepackage{amsmath,amssymb,geometry}
\geometry{margin=1in}
\title{Quantum Coherence Cosmology (QCC): Historical Evolution and Model Overview}
\author{Devin Lavrisha}
\date{120th Anniversary of Einstein's Unified Theory Quest}
\begin{document}
\maketitle

\section*{Overview}
Quantum Coherence Cosmology (QCC) is a unified field framework that reconstructs the fundamental structure of the universe using two causal, interacting coherence fields. The model evolves from early experimental RMS residual fitting to a full, dataset-validated theory encompassing General Relativity (GR), Quantum Field Theory (QFT), and cosmological observations. This document outlines the history of QCC's development.

\section*{Checkpoint Evolution}
\subsection*{Checkpoint V2.1: Wavelet-Sourced Normalization}
Used CMB Planck TT data mapped to redshift via gimbal transformation. Built $\phi(z)$, $\mu(z)$, and $\nu(z)$ fields from filtered harmonic modes. Achieved RMS < 1 across Pantheon+, DR9Q, KiDS.

\subsection*{Checkpoint V2.2A: Dual Coherence Canonicalization}
Introduced dual $\phi$ bundles: ``u-wells'' (matter sinks) and ``u-geysers'' (coherence eruptions). Reproduced cosmological structure without $\Lambda$CDM. Removed need for tunable parameters through structured envelope damping.

\subsection*{QFT Bridge Checkpoint: Canonical Lagrangian}
Formulated:
\[ \mathcal{L} = -\frac{1}{2} (\partial_t \phi)^2 + \frac{1}{2} (\nabla \phi)^2 - \frac{1}{2} m^2 \phi^2 - \alpha \phi^4 \]
Encoded mass, spin, and decay as emergent coherence behavior. Derivation supports curvature through stress-energy tensor $T^{\mu\nu}$.

\section*{Key Milestones}
\begin{itemize}
\item Coherence Mass Emergence: Matched electron, muon, pion, and neutron MeV via resonance bundling
\item Spin State Extraction: Angular momentum from $L_z = \int r \times (\nabla \phi) \, d^3x$
\item Hydrogen Atom Simulated: Proton (uud $\phi$-bundles) + Electron (toroidal shell)
\item Neutrinos Constructed: Long-wave coherent nodes with minimal energy curvature
\item Gauge Symmetries Emerged: U(1), SU(2) recovered from phase groupings of $\phi$ doublets
\end{itemize}

\section*{Pitfalls and Resolutions}
\begin{tabular}{ll}
\textbf{Pitfall} & \textbf{Resolution} \\
Static tuning disrupted structure & Switched to dynamic $\phi(z, \tau)$ evolution \\
MeV overshoot from bundle compression & Introduced bounded coherence radius \\
Symmetric inversion failed to conserve spin & Modeled real interference cancellation for antiparticles \\
Early quark modeling unstable & Built from base leptons upward \\
Normalization drift & Used localized envelope decay for harmonics \\
\end{tabular}

\section*{What Makes QCC Different}
\begin{itemize}
\item No dark matter, no $\Lambda$, no post hoc constants
\item Every particle and structure is emergent from field geometry
\item The only ingredients: memory field $\phi(z, \tau)$ and its decay curvature
\item Verified using real cosmological datasets
\end{itemize}

\section*{Current Capabilities}
\begin{itemize}
\item Electron, muon, tau, all neutrinos
\item Up, down, strange, charm, bottom, top quarks
\item Protons, neutrons
\item Hydrogen atom
\item Reconstructed SU(2)/U(1) gauge group
\item Early GR coupling via $T^{\mu\nu}$ from $\phi$ stress-energy
\end{itemize}

QCC is now a causal, reproducible, and data-anchored candidate for the most complete unification of physics yet proposed, arising not from tuning, but from geometric memory encoded at the birth of spacetime.
\end{document}
