\documentclass[12pt]{article}
\usepackage{amsmath,amssymb,amsfonts,graphicx,geometry,hyperref,booktabs}
\geometry{margin=1in}
\title{Quantum Coherence Cosmology (QCC)\Full Theory Checkpoint and Reconstruction Log}
\date{\today}
\begin{document}
	\maketitle
	
	\section*{\textcolor{blue}{Document Purpose}}
	This document serves as the \textbf{complete, reproducible record} of the Quantum Coherence Cosmology (QCC) model, from its raw data origins to its mathematical formalism, physical interpretation, simulation results, and Quantum Field Theory (QFT) integration. It includes all derivations, logic decisions, and computational methods.
	
	\textbf{Goal:} To allow any future physicist or reviewer to rebuild, verify, and peer review the QCC model step-by-step.
	
	\section{Origin Motivation}
	\textbf{Problem:} $\Lambda$CDM lacks a causal microphysical mechanism for:
	\begin{itemize}
		\item Dark matter
		\item Dark energy
		\item Coherent predictive structure formation
	\end{itemize}
	\textbf{Hypothesis:} These phenomena arise from \textbf{coherence memory fields} seeded during the CMB epoch and echoing through spacetime.
	
	\section{Raw Dataset Source}
	\textbf{Primary Input:} Planck PCCS 030 GHz Compact Source Catalog
	\begin{itemize}
		\item Columns used: RA, PSFFLUX
		\item Preprocessing:
		\begin{itemize}
			\item Remove NaNs and extreme outliers
			\item Convert to: $F(\text{RA}) = \log_{10}(\text{PSFFLUX} + 1)$
		\end{itemize}
	\end{itemize}
	\textbf{Why 030 GHz?} Near the CMB peak () and minimizes foreground contamination while preserving structure.
	
	\section{Mapping CMB to Coherence Geometry}
	\textbf{Step 1:} FFT of Flux vs RA\newline
	We interpret the CMB flux anisotropies as manifestations of underlying coherence modes. FFT extracts periodic structures interpreted as standing quantum wave harmonics across the RA slice.
	
	\textbf{Step 2:} Rescale RA to 720$^\circ$ Gimbal Domain:
	
	
	\textbf{Step 3:} Angular to Redshift Conversion (QCC Kernel):
	
	
	\section{Constructing $\phi(z)$}
	\begin{itemize}
		\item Maxima $\rightarrow \mu(z)$ (coherence bursts)
		\item Minima $\rightarrow \nu(z)$ (coherence wells)
	\end{itemize}
	Create $\delta$-functions at each $z$ node and apply Daubechies-4 wavelet transform:
	
	Redshift domain is log-stretched over $z \in [0.001, 10]$ to enable cross-dataset fitting.
	
	\section{Classical Field Theory Derivation}
	
	
	Potential forms:
	\begin{itemize}
		\item $V(\phi) = \Lambda (1 - \cos(\alpha \phi))$
		\item $V(\phi) = \sum A_i e^{-(z - z_i)^2 / 2\sigma^2}$
	\end{itemize}
	
	\section{Quantum Field Theory Structure}
	
	
	
	\section{Field Equations of QCC}
	To derive the governing equation for coherence propagation in redshift, we take variation over the action:
	
	Leading to the Euler–Lagrange equation for QCC:
	
	In redshift space, this becomes:
	
	This is solved via numerical integration using φ-burst initial conditions extracted from wavelet transformation. In the Gaussian well formalism:
	
	
	\section{Coherence Kernel & Causality}
	
	
	\textbf{Note:} $\lambda$ is not fitted, but emergent from local coherence decay, modulated by $\phi''(z)$ indicating geometry stability.
	
	\section{Dataset Application & Prediction Logic}
	\begin{itemize}
		\item \textbf{DR9Q:} $\phi'^2$ trace disruptions, RMS $\approx 0.29$
		\item \textbf{KiDS:} $\phi \rightarrow \xi_+$ via metric distortion
		\item \textbf{Pantheon+:} $\phi(z)$ modulates $H(z)$, SN lag
		\item \textbf{BAO:} $\phi(z)$ subtly shifts acoustic peak spacing
		\item \textbf{GWTC:} No $\phi$ match, likely local origin
	\end{itemize}
	
	\begin{table}[h!]
		\centering
		\begin{tabular}{lccc}
			\toprule
			Dataset & RMS Residual & Pearson $r$ & $p$-value \
			\midrule
			Pantheon+ & 0.3539 & -0.39 & $2.6 \times 10^{-13}$ \
			KiDS $\xi_+$ & 0.3837 & 0.57 & 0.11 \
			DR9Q & 0.4536 & -0.38 & $<10^{-17}$ \
			\bottomrule
		\end{tabular}
		\caption{QCC statistical validation across datasets.}
	\end{table}
	
	\section{Echo Cosmology Framework}
	All datasets are modeled as echo responses:
	
	Observables sample delayed geometric memory — not $\phi(z)$ directly.
	
	\section{Microphysical Bridge}
	\begin{itemize}
		\item $\phi''(x) < 0$ defines coherence wells
		\item Lab-scale testable energy injection:
		
		\item Predicts coherence traps, levitation, radiation capture
	\end{itemize}
	
	\section{Uncertainty Modeling}
	
	Uncertainties from flux propagate through wavelet transform. Apply Monte Carlo over $\phi(z)$ with observed flux error distributions to validate model confidence bounds.
	
	\section{CMB–Wavelet Overlay Validation}
	To verify memory extraction:
	\begin{itemize}
		\item Compare raw Planck TT spectrum with reconstructed $\phi(z)$ harmonics
		\item Align harmonic peaks and coherence bursts
	\end{itemize}
	
	\section{Summary}
	QCC is:
	\begin{itemize}
		\item A dynamic, predictive model
		\item Built from wavelet-extracted CMB structure
		\item Replacing $\Lambda$CDM with $\phi(z)$ coherence fields
		\item Compatible with QFT and lab-scale testing
		\item Reproducible from raw Planck data through simulation
	\end{itemize}
	It is ready for peer review, replication, and experimental exploration.
	
\end{document}

