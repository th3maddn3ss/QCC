\documentclass[12pt]{article}
\usepackage{amsmath, amssymb, graphicx, hyperref, authblk}
\usepackage[margin=1in]{geometry}
\title{Quantum Coherence Cosmology (QCC): A Predictive Framework for Replacing Dark Matter and Dark Energy}
\author{Devin Lavrisha}
\date{\today}

\begin{document}
	
	\maketitle
	
	\begin{abstract}
		Quantum Coherence Cosmology (QCC) replaces dark matter and dark energy with a dynamic scalar coherence field \( \phi(z, \tau) \) derived from cosmic microwave background (CMB) harmonics. Using wavelet decomposition of the Planck PCCS 030 GHz dataset, this model demonstrates predictive accuracy across datasets without tuning. We define a canonical Lagrangian anchored in quantum field theory, preserving microcausality and consistency with general relativity. RMS validation is performed across Pantheon+, KiDS, DR9Q, GWTC, and BAO datasets.
	\end{abstract}
	
	\tableofcontents
	
	\section{Introduction}
	The \textit{Quantum Coherence Cosmology} (QCC) framework proposes that apparent gravitational anomalies, typically attributed to dark matter and dark energy, are emergent effects of large-scale quantum coherence. This model constructs a scalar field \( \phi(z, \tau) \) that captures memory echoes seeded by early-universe fluctuations. Unlike phenomenological dark components, QCC grounds its dynamics in coherence field evolution and observable CMB imprinting, making it a physically consistent and data-anchored alternative.
	
	\section{Theoretical Foundations}
	\subsection{Origin of \( \phi(z) \)}
	Using a 720\textdegree{} gimbal FFT domain, we extract CMB harmonic modes from Planck PCCS 030 GHz data. These modes are mapped to redshift via a logarithmic transformation:
	\[ z = \frac{1100}{\ell + \varepsilon} \]
	Applying Daubechies-4 wavelets to isolate peak and trough bands, we define the fields:
	\[ \mu(z) = \text{Wavelet sum over peaks}, \quad \nu(z) = \text{Wavelet sum over troughs} \]
	The canonical field is:
	\[ \phi(z) = \mu(z) - \nu(z) \]
	This formulation preserves peak-trough quantum geometry, acting as a projection of coherence memory.
	
	\subsection{Canonical Lagrangian and Echo Term}
	The Lagrangian density for \( \phi(z, \tau) \) is:
	\[ \mathcal{L}_{\phi} = \frac{1}{2} g^{\mu\nu} \partial_\mu \phi \, \partial_\nu \phi - V(\phi) \]
	Where the potential includes:
	\[ V(\phi) = \frac{1}{2} m^2 \phi^2 + \frac{\lambda}{4} \phi^4 + \alpha e^{-\beta z} \phi(z, \tau) \]
	The echo kernel \( \alpha e^{-\beta z} \phi(z, \tau) \) captures coherence decay over redshift, tuned empirically from $\phi$-based residual minimization and gravitational lensing alignment.
	
	\section{Model Construction}
	\subsection{Wavelet Extraction and Mapping}
	Each CMB harmonic contributes a Gaussian pulse in redshift space. Peak envelopes form \( \mu(z) \), troughs form \( \nu(z) \). These are smoothed and summed, maintaining consistent phase structure.
	
	\subsection{Full Field Equations}
	\begin{align*}
		G_{\mu\nu} &= 8\pi G \left(T_{\mu\nu} + H_{\mu\nu}(\phi) \right) \\
		\nabla^{\mu} \left(T_{\mu\nu} + H_{\mu\nu}(\phi)\right) &= 0
	\end{align*}
	\( H_{\mu\nu}(\phi) \) is constructed from the stress-energy of the scalar field, preserving conformal symmetry and ensuring QFT compatibility.
	
	\section{Dataset Validation}
	\subsection{Pantheon+ Supernovae}
	Filtered for fit probability and uncertainty: \textasciitilde1048 high-quality entries. RMS residual \( \sim 0.35 \)
	\subsection{KiDS Weak Lensing}
	\( \xi_+(z) \) derived using lensing projection of \( \phi'(z)^2 \). RMS \( \sim 0.38{-}0.48 \)
	\subsection{DR9Q Quasar Redshifts}
	Binned histogram tested against \( \phi(z) \) structure. RMS \( \sim 0.45 \)
	\subsection{GWTC and BAO}
	Used for testing redshift drift and local coherence clustering against gravitational wave event fields.
	
	\section{Dark Matter Analog Test}
	\subsection{Halo Reconstruction from \( \phi \)}
	Local curvature induced by \( \phi'(z) \phi''(z) \) reconstructs shear profiles consistent with galaxy rotation curves. These coherence wells mimic gravitational halos without requiring exotic matter.
	
	\subsection{Comparison to \( \Lambda \)CDM}
	Unlike \( \Lambda \)CDM, QCC does not require parametric tuning. Shear, lensing, and expansion follow from coherence field topology.
	
	\section{Echo Kernel and Coherence Dynamics}
	\subsection{Echo Term Effects}
	\( \alpha e^{-\beta z} \phi(z, \tau) \) introduces a time-aware rebound kernel. Acts as coherence memory restoring force.
	
	\subsection{Microcausality}
	\( [\phi(x), \phi(x')] = 0 \) for spacelike separation, preserving QFT expectations.
	
	\subsection{Entropy and Information}
	Coherence decay satisfies second law:
	\[ S \propto \lambda_{\text{decay}} (\tau - \tau_0) \]
	This aligns with CFT models for quantum field entropy growth.
	
	\section{Publishing Readiness and Toolkit Structure}
	\subsection{GitHub Structure}
	\begin{verbatim}
		QCC Repository/
		- Codebase/
		- Datasets/
		- COM_PCCS_030_R2.04.txt
		- QCC_V2_Theory.tex
		- README.md
	\end{verbatim}
	
	\subsection{Next Steps}
	\begin{itemize}
		\item Package toolkit for CAMB-like integration
		\item Build \( \phi(z, \tau) \)-driven structure formation simulator
		\item Prepare Zenodo and arXiv submission
	\end{itemize}
	
	\appendix
	\section{Field Definitions and Constants}
	\begin{align*}
		\phi(z) &= \mu(z) - \nu(z) \\
		V(\phi) &= \frac{1}{2} m^2 \phi^2 + \frac{\lambda}{4} \phi^4 + \alpha e^{-\beta z} \phi(z, \tau)
	\end{align*}
	
	\section{Parameter References}
	\begin{itemize}
		\item \( m \): FFT-extracted envelope mass scale
		\item \( \lambda \): self-interaction from structure formation
		\item \( \alpha, \beta \): fitted echo decay coefficients
	\end{itemize}
	
	\section{Dataset Preprocessing Notes}
	\textbf{Pantheon+}: culled for poor fit probability and excess uncertainty.  
	\textbf{KiDS}: $\xi_+$ extracted, normalized, smoothed.  
	\textbf{DR9Q}: binned redshift histogram normalized against $\phi$(z) span.  
	\textbf{GWTC/BAO}: aligned temporally to $\phi$(z, $\tau$) predictions.
	
\end{document}
