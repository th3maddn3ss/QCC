
\documentclass[12pt]{article}
\usepackage{amsmath, amssymb, geometry, hyperref}
\geometry{margin=1in}
\title{QCC V2.2A Toolkit Usage and Reproducibility Guide}
\author{Devin Lavrisha}
\date{May 2025}

\begin{document}
\maketitle

\section*{Abstract}
This document serves as a practical guide for researchers to reproduce the Quantum Coherence Cosmology (QCC) V2.2A model using the provided toolkit. It outlines installation requirements, dataset expectations, kernel projection steps, and RMS/statistical validation, ensuring full transparency and reproducibility of the published results.

\section{Toolkit Overview}
The QCC model is implemented in \texttt{QCC\_toolkit\_V2.1\_fixed.py}, which defines a dynamic scalar coherence field $\phi(z, \tau)$ and its projected kernel $K(z)$. This kernel is compared to redshift-distributed cosmological data to validate coherence-based structure formation.

\subsection*{Class Structure}
\begin{itemize}
\item \texttt{QCCModel}: Initializes parameters, integrates $\phi(z, \tau)$ via Runge-Kutta.
\item \texttt{project\_kernel(z)}: Projects the normalized kernel onto dataset redshift bins.
\item \texttt{get\_kernel\_data()}: Returns raw $\phi$, $\phi'$, $\phi''$, and $K(z)$.
\end{itemize}

\section{Installation Requirements}
\begin{itemize}
\item Python 3.8+
\item NumPy
\item SciPy
\item PyWavelets
\item Matplotlib (optional)
\end{itemize}

Install with:
\begin{verbatim}
pip install numpy scipy pywt matplotlib
\end{verbatim}

\section{Dataset Format}
Ensure cleaned CSV files are provided with these columns:
\begin{itemize}
\item \textbf{Pantheon+}: redshift (\texttt{redshift}), magnitude (\texttt{mB})
\item \textbf{DR9Q}: redshift bin center (\texttt{z\_bin\_center}), quasar count
\item \textbf{KiDS}: angle (\texttt{theta\_arcmin}), correlation (\texttt{xi\_plus})
\item \textbf{GWTC}: redshift bin center, merger count
\end{itemize}

\section{Kernel Projection \& RMS Calculation}
\begin{enumerate}
\item Load the toolkit and model:
\begin{verbatim}
from QCC_toolkit_V2_1_fixed import QCCModel
model = QCCModel(mode='dynamic')
\end{verbatim}
\item Project kernel onto redshift bins:
\begin{verbatim}
kernel_proj = model.project_kernel(dataset['z_column'])
\end{verbatim}
\item Compute RMS:
\begin{verbatim}
from sklearn.metrics import mean_squared_error
rms = np.sqrt(mean_squared_error(normalized_data, kernel_proj))
\end{verbatim}
\end{enumerate}

\section{Statistical Evaluation}
The toolkit supports:
\begin{itemize}
\item Root Mean Squared Error (RMS)
\item Akaike Information Criterion (AIC)
\item Bayesian Information Criterion (BIC)
\item Chi-Squared test ($\chi^2$)
\item Kolmogorov–Smirnov test (KS)
\end{itemize}

\section{Expected Output Summary}
\begin{tabular}{l|c}
\textbf{Dataset} & \textbf{Expected RMS} \\
\hline
Pantheon+ & 0.215 \\
DR9Q      & 0.568 \\
KiDS      & 0.547 \\
GWTC      & 0.527 \\
\end{tabular}

\section{Troubleshooting}
\begin{itemize}
\item \textbf{NaN output}: check redshift range.
\item \textbf{Flat kernel}: verify wavelet normalization step.
\item \textbf{Shape mismatch}: align redshift array to projected output.
\end{itemize}

\section*{Final Note}
This document should accompany the core model, physics validation, and statistical analysis for full reproducibility of QCC V2.2A.

\end{document}
