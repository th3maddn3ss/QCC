
\documentclass[12pt]{article}
\usepackage{amsmath,amsfonts,graphicx,hyperref}
\usepackage[margin=1in]{geometry}
\title{On the Responsible Use of LLMs in Physics: Assistants, Not Architects}
\author{Devin Lavrisha}
\date{August 2025}

\begin{document}
\maketitle

\section*{Abstract}
Recent debates have highlighted concerns regarding the rise of so-called "vibe physics" and the use of large language models (LLMs) to generate physics theories from scratch. In this short position paper, we respond directly to critiques outlined in arXiv:2507.06952v2, emphasizing the importance of using LLMs as assistants in scientific discovery, not as primary generators of theoretical frameworks. We demonstrate, using the Quantum Coherence Cosmology (QCC) project as a working example, that LLMs can support—but not supplant—rigorous, empirically anchored scientific work.

\section{The Central Critique}
The paper \textit{"AI-Slops in Physics and the Crisis of Meaning"} (arXiv:2507.06952v2) articulates a growing concern: that LLMs are being used not as tools but as autonomous theorists. The danger lies in conflating symbolic coherence with physical causality, and mistaking aesthetic symmetry for empirical truth.

\section{The Role of LLMs as Assistants}
LLMs excel at:
\begin{itemize}
  \item Translating natural language into executable code
  \item Debugging and refactoring computational pipelines
  \item Extracting mathematical structure from natural descriptions
  \item Rapidly testing and iterating numerical simulations
\end{itemize}

However, they lack:
\begin{itemize}
  \item Causal awareness (they do not inherently distinguish cause from effect)
  \item Grounding in physical datasets unless explicitly provided
  \item The ability to generate falsifiable hypotheses without external constraints
\end{itemize}

\textbf{Thus, responsibility for causal logic, empirical anchoring, and theoretical rigor remains solely with the human researcher.}

\section{Case Study: QCC as a Structured Human-Led Framework}
In the Quantum Coherence Cosmology (QCC) framework:
\begin{itemize}
  \item All data is sourced from observational cosmology (Planck CMB, Pantheon+, KiDS, etc.)
  \item The Lagrangian and scalar field model \( \phi(z, \tau) \) are human-derived, based on variational principles
  \item Code for simulations and projections was assisted by LLMs under direct instruction, but never generated spontaneously
  \item No synthetic values or hallucinated constants are introduced—every parameter is either derived, observed, or empirically fit with full traceability
\end{itemize}

\section{On Causality and the Limits of LLMs}
Causality is not a constraint LLMs learn—rather, they model patterns in text. A theory generated by an LLM cannot be assumed to preserve causal structure unless explicitly imposed. In QCC and similar efforts, it is the human's role to:
\begin{enumerate}
  \item Define physical laws and conservation principles
  \item Verify that equations preserve microcausality and relativistic consistency
  \item Validate outputs against observational data
\end{enumerate}

Failing to do so risks creating "physics-shaped" models that collapse under scrutiny.

\section{Conclusion}
As the field of theoretical physics explores integration with AI tools, we must draw a firm line: LLMs are assistants, not architects. Used responsibly, they are accelerators of human insight. Used carelessly, they generate noise masked as elegance. The future of cosmology depends on maintaining this distinction.

\vspace{1em}
\noindent\textbf{Citation:} Lavrisha, D. \textit{On the Responsible Use of LLMs in Physics: Assistants, Not Architects}. (2025). Commentary inspired by arXiv:2507.06952v2.

\end{document}
